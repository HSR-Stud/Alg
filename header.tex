% Genereller Header
\documentclass[10pt,twoside,a4paper,fleqn]{article}
\usepackage[utf8x]{inputenc}
\usepackage[left=1cm,right=1cm,top=1cm,bottom=1cm,includeheadfoot]{geometry}
\usepackage[ngerman]{babel,varioref}


% Pakete
\usepackage{amssymb}
\usepackage{amsmath}
\usepackage{fancybox}
\usepackage{bm}       % Bold Math
\usepackage{graphicx}
\usepackage{color}
\usepackage{lastpage}
\usepackage{wrapfig}
\usepackage{fancyhdr}
\usepackage{hyperref}
\usepackage{tabularx} % better tables
%\usepackage{verbatim}
%\usepackage{floatflt}
\usepackage{multirow} % zellen in tabellen verbinden
\usepackage{multicol} 

%\usepackage{arydshln}
\usepackage{ucs}
\usepackage{pdflscape} % landscape
%\usepackage{slashbox} % getrennte zelle in tabelle
%\usepackage{array} % anordnung in tabellen
%Show pdf in 100 zomm factor!
\usepackage{hyperref}
\usepackage{listings}               %Programmcodeumgebung für Matlab Code

%%%%%%%%%%%%%%%%%%%%
% Generelle Makros %
%%%%%%%%%%%%%%%%%%%%
\newcommand{\skript}[1]{$_{\textcolor{red}{\mbox{\small{Skript S. #1}}}}$}
\newcommand{\sachs}[1]{$_{\textcolor{blue}{\mbox{\small{Sachs S. #1}}}}$}
\newcommand{\formelbuch}[1]{$_{\textcolor{red}{\mbox{\small{S#1}}}}$}
\newcommand{\hayes}[1]{$_{\textcolor{red}{\mbox{\small{Hayes p #1}}}}$}
\newcommand{\verweis}[2]{ {\small (siehe auch \ref{#1}, #2 (S. \pageref{#1}))}}
\newcommand{\subsubadd}[1]{\textcolor{black}{\mbox{#1}}}
\newenvironment{liste}[0]{\begin{list}{$\bullet$}{\setlength{\itemsep}{0cm}\setlength{\parsep}{0cm} \setlength{\topsep}{0cm}}}{\end{list}}
    
\newcommand{\logd}[0]{\log_{10}}
\newcommand{\subsubsubsection}[1]{\textbf{#1}}
\newcommand{\matlab}[1]{\footnotesize{(Matlab: \texttt{#1})}\normalsize{}}

\newenvironment{aufzaehlung}[0]{\begin{enumerate}{\setlength{\itemsep}{0cm}\setlength{\parsep}{0cm}\setlength{\topsep}{0cm}}} {\end{enumerate}}

\newcommand{\abbHeight}[3]{
	\begin{center}
		\includegraphics[height=#2]{./bilder/#1} \\
		#3
    \end{center}
}

\newcommand{\todo}[1]{\colorbox{red}{#1}}
%\newcommand{\skriptsection}[2]{\section{#1 {\tiny Skript S. #2}}}
%\newcommand{\skriptsubsection}[2]{\subsection{#1 {\tiny Skript S. #2}}}
%\newcommand{\skriptsubsubsection}[2]{\subsubsection{#1 {\tiny Skript S. #2}}}
%\renewcommand{\skriptsection}[2]{\section{#1 {\tiny Schaum S. #2}}}
%\renewcommand{\skriptsubsection}[2]{\subsection{#1 {\tiny Schaum S. #2}}}
%\renewcommand{\skriptsubsubsection}[2]{\subsubsection{#1 {\tiny Schaum S. #2}}}
\newcommand{\skriptsection}[2]{\section{#1 \formelbuch{#2}}}
\newcommand{\skriptsubsection}[2]{\subsection{#1 \formelbuch{#2}}}
\newcommand{\skriptsubsubsection}[2]{\subsubsection{#1 \formelbuch{#2}}}

%%%%%%%%%%
% Farben %
%%%%%%%%%%
\definecolor{black}{rgb}{0,0,0}
\definecolor{red}{rgb}{1,0,0}
\definecolor{white}{rgb}{1,1,1}
\definecolor{grey}{rgb}{0.8,0.8,0.8}

%%%%%%%%%%%%%%%%%%%%%%%%%%%%
% Mathematische Operatoren %
%%%%%%%%%%%%%%%%%%%%%%%%%%%%
\DeclareMathOperator{\sinc}{sinc}
\DeclareMathOperator{\sgn}{sgn}
\DeclareMathOperator{\Real}{Re}
\DeclareMathOperator{\Imag}{Im}
\DeclareMathOperator{\grad}{grad}
\DeclareMathOperator{\rank}{rank}



% Fouriertransformationen
\unitlength1cm
\newcommand{\FT}
{
\begin{picture}(1,0.5)
\put(0.2,0.1){\circle{0.14}}\put(0.27,0.1){\line(1,0){0.5}}\put(0.77,0.1){\circle*{0.14}}
\end{picture}
}


\newcommand{\IFT}
{
\begin{picture}(1,0.5)
\put(0.2,0.1){\circle*{0.14}}\put(0.27,0.1){\line(1,0){0.45}}\put(0.77,0.1){\circle{0.14}}
\end{picture}
}
\newcommand{\jw}{j\omega}

\newcommand{\numbercircled}[1]{\textcircled{\raisebox{-1pt}{#1}}}

\newcommand{\DFT}
{
%\overset{DFT}{
	\begin{picture}(1,0.2)
	\put(0.2,0.1){\circle{0.14}}{\put(0.27,0.1){\line(1,0){0.5}}}\put(0.77,0.1){\circle*{0.14}}
	\end{picture}
%}
}

\newcommand{\IDFT}
{
%\overset{IDFT}{
    \begin{picture}(1,0.2)
	\put(0.2,0.1){\circle*{0.14}}\put(0.27,0.1){\line(1,0){0.45}}\put(0.77,0.1){\circle{0.14}}
	\end{picture}
%}
}
\newcommand{\partFrac}[2]{\frac{\partial #1}{\partial #2}}

%%%%%%%%%%%%%%%%%%%%%%%%%%%%
% Code Listings
%%%%%%%%%%%%%%%%%%%%%%%%%%%%
\lstloadlanguages{Matlab} 
\lstnewenvironment{MatlabCode}[1][] 
{
\lstset{ 
    language= Matlab, 
    basicstyle=\ttfamily,                     %generell Schreibmaschinenschrift 
    basicstyle=\scriptsize, 
    showstringspaces=false,                  %In Strings keine Backspace zeichen breaklines=true, 
    numbers   =   left,                              %links Zeilennummern 
    xleftmargin=.04\textwidth, 
    %frame=single,                                 %shadowbox, leftline, lines, topline, t, r, b, l 
    #1} 
} 
{}


%%%%%%%%%%%%%%%%%%%%%%%%%%%%
% Allgemeine Einstellungen %
%%%%%%%%%%%%%%%%%%%%%%%%%%%%
%pdf info
\hypersetup{pdfauthor={\authorinfo},pdftitle={\titleinfo},colorlinks=false}
\author{\authorinfo}
\title{\titleinfo}
\hypersetup{pdfstartview={XYZ null null 1.00}}

%Kopf- und Fusszeile
\pagestyle{fancy}
\fancyhf{}
%Linien oben und unten
\renewcommand{\headrulewidth}{0.5pt} 
\renewcommand{\footrulewidth}{0.5pt}


\fancyhead[L]{\titleinfo{ }- Summary}
%Kopfzeile rechts bzw. aussen
\fancyhead[R]{\today{ }- Page \thepage/\pageref{LastPage}}
\fancyfoot[C]{\copyright{ }\authorinfo}

% Einr�cken verhindern versuchen
\setlength{\parindent}{0pt}

